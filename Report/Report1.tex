\documentclass[twocolumn,letterpaper]{report}
\usepackage[utf8]{inputenc}
\usepackage{titlesec}
\usepackage{geometry}
\usepackage{multirow}
\usepackage{graphicx}
\graphicspath{ {Images/} }
\usepackage{subfig}
\usepackage{amsmath}
\usepackage{hyperref}
\usepackage{array}
\usepackage[none]{hyphenat} %To avoid split lines at the end of the line (divisione in sillabe)
\usepackage{enumitem}
\usepackage{dcolumn} % automatically loads the 'array' package
\geometry{
	 a4paper,
	 total={210mm,297mm},
	 left=8mm,
	 right=8mm,
	 top=20mm,
	 bottom=15mm
 }

%For code pasting
\usepackage{listings}
\usepackage{color}

\definecolor{dkgreen}{rgb}{0,0.6,0}
\definecolor{gray}{rgb}{0.5,0.5,0.5}
\definecolor{mauve}{rgb}{0.58,0,0.82}

\usepackage{setspace}

\lstset{frame=tb,
  language=Python,
  aboveskip=1mm,
  belowskip=1mm,
  showstringspaces=false,
  columns=flexible,
  basicstyle={\small\ttfamily},
  numbers=left,
  numberstyle=\tiny\color{gray},
  keywordstyle=\color{blue},
  commentstyle=\color{dkgreen},
  stringstyle=\color{mauve},
  breaklines=true,
  breakatwhitespace=true,
  tabsize=3
}

\newcounter{debug}
\setcounter{debug}{2}

%https://tex.stackexchange.com/questions/291307/how-to-hide-show-section-levels-in-the-table-of-contents
\setcounter{tocdepth}{3}
\titleformat{\chapter}[display]{\normalfont\bfseries}{}{0pt}{\Huge}
%Space before and after \chapter {left}{before}{after}
\titlespacing*{\chapter}{0pt}{-50pt}{1pt}
\titlespacing*{\section}{0pt}{8pt}{1pt}
\titlespacing*{\subsection}{0pt}{8pt}{1pt}
\titlespacing*{\subsubsection}{0pt}{8pt}{1pt}

%Separation between the columns (2 columns report)
\setlength{\columnsep}{15pt}
%Space after a `space'
\setlength{\parskip}{0mm}


\begin{document}
%Space between elements of a list use (nosep for no space between items) {itemsep=-6mm}
\setlist[itemize]{nosep,topsep={1pt},partopsep={0pt}}

%Remove spaev after and before \equation
\setlength{\abovedisplayskip}{0pt}
\setlength{\belowdisplayskip}{3pt}
\setlength{\abovedisplayshortskip}{0pt}
\setlength{\belowdisplayshortskip}{3pt}

\begin{titlepage}
   \begin{center}
       %\vspace*{1cm}
       \Huge
       \textbf{POLITECNICO DI TORINO}
       \vspace{1.5cm}
       
       \includegraphics[scale=0.24]{logoPolito-1.png}
       %\vspace{0.8cm}
       
       \hspace{0pt}
					\vfill
					
					\Huge
        \textbf{Computer aided simulations and \\
        								performance evaluation}
       \vspace{1cm}
       
       \Large
       Academic year 2020/21
       \vfill
       
       \vfill
				\hspace{0pt}
       \begin{flushright}
       			\large
            VASSALLO Maurizio, s276961
       \end{flushright}
   \end{center}
\end{titlepage}

\onecolumn
\tableofcontents
\newpage
\clearpage

\twocolumn
% ### REPORT 1 ###
\ifnum\value{debug}=1 {
    
\chapter{Queue Systems}
	These experiments are performed to understand the behaviour of a queue system with a given number of servers (one or more), a capacity waiting line (finite or infinite) a FIFO (First In First Out) serving discipline and inter-arrivals time that follow a Poisson process.
	
	 \section{A single-Server queuing system}
	 			\subsection{Exponential Distributed Service Time}
	 						\includegraphics[width = 0.45\textwidth]{Lab5/QueueSystemExponentialDistributionInfiniteQueueSingleServer3Runst1.png} \\
								It is possible to see that the results obtained with simulation are close to the theoretical formula. 
			The theoretical value is calculated as:
			\begin{equation}
				theorical\_queue\_time = \frac{1}{ \mu - \lambda} 
			\end{equation}
				where $\lambda=\frac{1}{ARRIVAL}$ and $\mu=\frac{1}{SERVICE}$
			
	 			\subsection{Uniform Distributed Service Time}
	 			\includegraphics[width = 0.45\textwidth]{Lab5/QueueSystemUniformDistributionInfiniteQueueSingleServer3Runst1.png} \\
	 			\begin{equation}
						general\_th\_queue\_time = ro \times ( 1 + ro \frac{1+C_s^2}{2\times(1-ro)} )
				\end{equation}
						where $ro=\frac{\lambda}{\mu}=\frac{LOAD}{2}$, $C_s^2=\frac{Var(S)}{E^2(s)}$ is the \emph{Coefficient of Variation} (ratio between the Variance and the Mean squared) and it depends on the service time distribution. \\
						This is the general formula indeed can be applied also to a service time with Exponential distribution ($C_s^2=1$); in the case of Uniform distribution the $C_s^2=\frac{1}{3}$. \\
						It is possible to see that the average waiting time is lower with respect to the Exponential distribution. Therefore to improve the performance of a queue system a possible way is to assign to a job a service time that is uniformly distributed (not always possible to decide a priori).
						
						
						
	\section{Finite Capacity of the waiting line}
	 			\subsection{Exponential Distributed Service Time}
	 						\includegraphics[width = 0.45\textwidth]{Lab5/QueueSystemExponentialDistributionFiniteQueueSingleServer3Runst2.png} \\
It is possible to see that the average waiting time converges for any value of \emph{B} to a specific value. This value is $queue\_time = B \times SERVICE$, where B is the maximum value of allowed client in the system and SERVICE is the average time spent in a server. \\
It is worth to mention that with an unlimited system length capacity the average waiting time would diverge for any $LOAD \ge 1$.

	 					\includegraphics[width = 0.45\textwidth]{Lab5/QueueSystemExponentialDistributionFiniteQueueSingleServer3Runst2_probloss.png} \\
	 					The probability of loosing a customer tends to 1 as the load increases.
	 					
	 			\subsection{Uniform Distributed Service Time}
	 			\includegraphics[width = 0.45\textwidth]{Lab5/QueueSystemUniformDistributionFiniteQueueSingleServer3Runst2.png} \\
	 			Also for the Uniform distribution service time, the average waiting time converges for any value of \emph{B} to a specific value. This value is $queue\_time = B \times\frac{SERVICE}{2}$ since the mean value of a Uniform distribution (U(A,B)) is $E(x)=\frac{(B-A)}{2}$; in this case A is 0 and B is the SERVICE time. \\
	 			With an unlimited queue length the average waiting time would diverge for any $LOAD \ge 2$.
	 			
	 		\includegraphics[width = 0.45\textwidth]{Lab5/QueueSystemUniformDistributionFiniteQueueSingleServer3Runst2_probloss.png} \\
	 			For the Uniform distribution service time the probability of loss a customer tends to 1 but in a slower way with respect to the Exponential distribution.
	 			
	 			
	 			\section{Multiple-Server queuing system}
	 			\subsection{Random Policy}
	 			With the Random policy when a new customer arrives he chooses a random server among the ones that are not busy.
			 			\subsubsection{Exponential Distributed Service Time}
			 			\includegraphics[width = 0.45\textwidth]{Lab5/QueueSystemExponentialDistributionInfiniteQueueMultiServer3Runst3s1.png} \\
		It is possible to notice that the average waiting time, for a given load, decreases with increasing the number of available servers
		
			 			\subsubsection{Uniform Distributed Service Time}
			 		\includegraphics[width = 0.45\textwidth]{Lab5/QueueSystemUniformDistributionInfiniteQueueMultiServer3Runst3s1.png} \\
			 			The same happens for the Uniform Distribution.
			 			
			 	\subsection{Fastest Policy}
	 			With the Fastest policy when a new customer arrives he chooses the fastest server among the ones that are not busy.
			 			\subsubsection{Exponential Distributed Service Time}
			 			\includegraphics[width = 0.45\textwidth]{Lab5/QueueSystemExponentialDistributionInfiniteQueueMultiServer6Runst3s2.png} \\
		This policy seems to work not very well (with respect to the Random policy) with a low number of services while it performs slightly better if the number of services increases.
		
								
			 			\subsubsection{Uniform Distributed Service Time}
			 		\includegraphics[width = 0.45\textwidth]{Lab5/QueueSystemUniformDistributionInfiniteQueueMultiServer6Runst3s2.png} \\
			 			Also for the Uniform distributions the performance are better even if for some values it seems to diverge earlier (m=2, m=8).
			 			
	 			\subsection{Finite Queue}
			 			\subsubsection{Exponential Distributed Service Time}
			 			\includegraphics[width = 0.45\textwidth]{Lab5/QueueSystemExponentialDistributionFiniteQueueMultiServer3Runst3.png} \\
		Also for multi server, in the case of a finite queue system, the average waiting time converges to a defined value. This value is $queue\_time = B \times\frac{SERVICE}{m}$ (where m is 2 in the graph above), so it is better than a single server system.
		
			 			\subsubsection{Uniform Distributed Service Time}
			 		\includegraphics[width = 0.45\textwidth]{Lab5/QueueSystemUniformDistributionFiniteQueueMultiServer3Runst3.png} \\
			 			Similar for the Uniform distribution, it converges to $queue\_time = B \times\frac{SERVICE}{2\times m}$.
			 			
} \fi 



% ### REPORT 2 ###
\ifnum\value{debug}=2 {
    
\chapter{Bins \& Balls}
	 
	\section{Introduction}
	 
	This experiment involves a number \emph{N} of bins and balls: for each ball random bins are chosen, one or more than one, depending on the dropping policy; given those bins a ball is put in one of them.	The goal is to evaluate the maximum bins occupancy and compare the results with the theoretical ones. \\
	 There are 2 dropping policy:
	\begin{itemize}
		\item Random Dropping: for each ball a single random bin is chosen and the ball is put in it; 
		\item Random Load Balancing: for each ball \emph{d} random bins are chosen and the ball is put only in the one with the lowest occupancy. In this simulation the values of \emph{d} used are 2 and 4.
	\end{itemize}

	\section{Tasks}
	 
		\subsection{Input Parameters}
			The input parameters of the simulation are:
			\begin{itemize}
				\item Number N of bins and balls;
				\item The seed value used to initialize a pseudorandom number generator;
				\item The confidence level used to calculate the confidence interval;
				\item The number of runs: the number of times that we run our simulation. This is done in order to have more accurate results.
			\end{itemize}
			 
	\subsection{Output Metrics}
			The output metrics of the simulation are:
			\begin{itemize}
				\item Number N of bins and balls;
				\item The lower confidence interval value;
				\item The average max occupancy value;
				\item The upper confidence interval value;
				\item The relative error.
			\end{itemize}
			All these value are stored in a file and each field is tab separated.
	
	\subsection{Main data structure}
			The data structure used is a numpy array of integers. This array has length $num\_bins$ and it stores the bins occupancy. This data structure allows to have a constant access time ($\mathcal{O}(1)$) and also a worst-case constant access time ($\mathcal{O}(1)$).
			
	\subsection{Simulator Inputs \& Outputs}
			The whole simulation runs inside a script where the simulator runs multiple times for multiple values of bins and balls. \\
			The output is a .dat file that contains the output metrics, therefore this file will contain a number of lines equal to the number of different values of bins and balls used. This .dat file is then elaborated by scripts in order to create some plots. There are 3 scripts for plotting:
			\begin{itemize}
				\item One plots the results of the simulation in order to have a comparison with the theoretical occupancy values;
				\item One plots the different performances of the dropping policy;
				\item One plots the relative errors for different values of the number of runs.
			\end{itemize}
			
			\subsection{Coherence Simulation \& Theoretical Formula}
			
			\noindent
			
			\includegraphics[width = 0.45\textwidth]{Lab1/RandomDroppingPolicyRuns3.png} \\
			It is possible to see that 3 runs are not enough since the confidence interval is not inside the theoretical values. \\
			The lower bound is calculated using the formula:
			
			\begin{equation} \label{eq:1}
				 \text{expected\_max\_occupancy = }\frac{\log n}{\log \log n} 
			\end{equation}
			where \emph{n} is the number of bins and balls,\\
			while the upper bound is just 3 times this formula.
						
			\includegraphics[width = 0.45\textwidth]{Lab1/RandomDroppingPolicyRuns5.png} \\
			It is possible to see that 5 runs are enough since the confidence interval is inside the theoretical values.
						
			\includegraphics[width = 0.45\textwidth]{Lab1/RandomLoadBalancingd2Runs3.png} \\
			Even with 3 runs the Load Balancing with \emph{d=2} is near to the theoretical value.
			\begin{equation} \label{eq:2}
				 \text{expected\_max\_occupancy = }\frac{\log \log n}{\log d}
			\end{equation}
			Where \emph{n} is the number of bins and balls and \emph{d} the number of random bins chosen. \\
			It is possible to see that this formula is much smaller than formula \ref{eq:1}, so we expect a lower occupancy value.
			
			\includegraphics[width = 0.45\textwidth]{Lab1/RandomLoadBalancingd4Runs3.png} \\
			Even with 3 runs the Load Balancing with \emph{d=4} is quite near to the theoretical value. 
			It is possible to see that the average occupancy is lower with respect to the Load Balancing with \emph{d=2} (The 2 plots have the same y-axis scale).
			
			\includegraphics[width = 0.45\textwidth]{Lab1/ComparisonamongPolicies,Runs5.png} \\
			With this graph it is possible to see the differences between the policies and in particular: the Balancing policy works better than the Random one and that with an increasing number of bins selected the maximum occupancy decreases. \\ There must be a trade-off between the number of bins selected and the script execution time because with large values of \emph{d} the occupancy decreases but the execution time increases: in the extreme case with \emph{d=\#bins} we would put the ball in the least occupied bins (similar to execute \emph{np.argmin(bins)}) and have an average maximum occupancy of 1 (one ball in each bin)  but that would require some time, especially for large number of bins.
			
			\includegraphics[width = 0.45\textwidth]{Lab1/RelativeErrorsRandomDroppingPolicy.png} \\
			It is possible to see that the relative errors decrease with increasing the number of runs. 
			\begin{equation}\label{eq:3}
				 \text{rel\_error = }2\frac{\Delta}{x}
			\end{equation}
			\begin{center}
					where $\Delta$ is the half the \emph{CI} length.
			\end{center}
			Similar results are obtained with the Loading Balancing policy; graphs are omitted.
					
			
\chapter{Birthday Paradox}
	 
	 \section{Introduction}
	 
	This experiment involves a number \emph{m} people: for each person a random number, the birthday in the case of Birthday Paradox, is chosen among \emph{n} possible values. \\
	The goal is to evaluate:
	\begin{itemize}
		\item The probability of conflict. A conflict is experienced when two people have the same random number (same birthday);
		\item The minimum number of people required for a conflict.
	\end{itemize}

	\section{Tasks}
	 
		\subsection{Input parameters}
			The input parameters of the simulation are:
			\begin{itemize}
				\item Number of possible ``days'': in this simulation, this value can be: [365, $10^5$, $10^6$];
				\item The seed value used to initialize the pseudorandom number generator;
				\item The confidence level used to calculate the confidence interval;
				\item The number of runs: the number of times that we run our simulation. This is done in order to have more accurate results;
				\item A flag depending if we want to calculate the conflict probability or the minimum number of people needed to experience a conflict.
			\end{itemize}
			 
	\subsection{Output metrics}
			The output metrics of the simulation are:
			\begin{enumerate}
					\item Conflict Probability:
					\begin{itemize}
							\item Number N of persons;
							\item The lower confidence interval value;
							\item The average max occupancy value;
							\item The upper confidence interval value;
							\item The relative error.
					\end{itemize}
					\item Minimum number of people:
					\begin{itemize}
							\item The lower confidence interval value;
							\item The average number of people;
							\item The upper confidence interval value;
							\item The relative error;
							\item The theoretical value.
					\end{itemize}
			\end{enumerate}
			All these value are stored in a file and each field is tab separated.
	
	\subsection{Simulator main data structure}
				The data structure used is a numpy binary array to store whether the element (day) at position \emph{i} is occupied or not (in the case of Birthday Paradox, there is already, at least, one person who was born in that day). \\ 
				This data structure allows to have a constant access time ($\mathcal{O}(1)$) and also a worst-case constant access time ($\mathcal{O}(1)$). \\
				For the calculation of the probability there is a counter that keeps how many conflicts happened for a given number of people. This is used to calculate the probability as $prob(conf)=\frac{\#conflicts}{\#people}$ at each run. \\
				%For the minimum number of people there is a an array storing the minimum number of people for a conflict at each run.
			
	\subsection{Simulator inputs and outputs}
			The whole simulation runs inside a script. The simulator runs each time for different values of people or just one in the case of finding the minimum number of people for a conflict. \\
			At each iteration the output is a .dat file that contains the output metrics, therefore this file will contain a number of lines equal to the number of different values of people or just one line in the case of minimum case. For the first case, the .dat file is then elaborated by a script to create the plots.
		
		\subsection{Coherence simulation \& theoretical formula}
				\includegraphics[width = 0.45\textwidth]{Lab2/ProbabilityConflictDays365Runs250.png} \\
				\includegraphics[width = 0.45\textwidth]{Lab2/ProbabilityConflictDays100000Runs250.png} \\
				\includegraphics[width = 0.45\textwidth]{Lab2/ProbabilityConflictDays1000000Runs250.png} \\
				For $10^6$ case, even with 250 runs the simulation is not very close to the theoretical result. \\
				\includegraphics[width = 0.45\textwidth]{Lab2/ProbabilityConflictDays1000000Runs600.png} \\
				Better results can be obtained increasing the number of runs but such a huge number of runs requires more time to be executed.
				
		\subsection{Is the theoretical formula for the conflict probability accurate?}
				Yes, how it can be seen from the previous graphs, the formula is accurate but for large number of days it is less accurate and it requires an higher number of runs.
				\includegraphics[width = 0.45\textwidth]{Lab2/RelativeErrorDays100000Runs250.png} \\
				It is possible to see the accuracy of the theoretical formula through the relative errors plot that the they tend to 0.
				
				
		\subsection{Required elements given a priori probability}
				Yes, it is possible using the following formula:
				\[
						num\_elements \: = \: \sqrt{2 \times number\_days \times \log{ \left( \frac{1}{1-p} \right) } }
				\]
							where \emph{p} is the wanted probability, $ p \in [0,1)$. \\
							
		With this formula it is possible to find all values of \emph{m} given a value of \emph{p}, the only value non possible to find is \emph{p=1} since this is not allowed in the formula (division by 0) but we have $100\%$ probability of conflict if $number\_days+1$ elements (people) are chosen.
		
		\subsection{Summary results}
				\begin{table}[h]
				\renewcommand{\arraystretch}{1.05}
				  \resizebox{0.5\textwidth}{!}{%
							\centering
							%|l|l|l|l|l|l|
							\begin{tabular}{|c|c|c|c|c|c|c|}
							\hline
							\multicolumn{1}{|c|}{\textbf{n}} & 
							\multicolumn{1}{c|}{\textbf{\begin{tabular}[c]{@{}c@{}}num\\runs\end{tabular}}} & 
							\multicolumn{1}{c|}{\textbf{\begin{tabular}[c]{@{}c@{}}Avg num\\people for\\a conflict\end{tabular}}} & 
							\multicolumn{1}{c|}{\textbf{\begin{tabular}[c]{@{}c@{}}95\%\\CI\end{tabular}}} & 
							\multicolumn{1}{c|}{\textbf{\begin{tabular}[c]{@{}c@{}}Theo.\\ value\end{tabular}}} & 
							\multicolumn{1}{c|}{\textbf{\begin{tabular}[c]{@{}c@{}}MAE\end{tabular}}}\\ \hline
							%							days				runs								simul												CI						theor										MAE
							\textbf{365}           										 & 250 			& 22.73						& 1.44      & 23.94							& 1.21				\\ \hline
							\textbf{10\textsuperscript{5}}   &	250			& 401.72       & 27.13    	& 396.33      	& 5.39								\\ \hline
							\textbf{10\textsuperscript{6}}   &	250			& 1154.26     & 79.76				&	1253.31      & 99.05							\\ \hline
							\textbf{10\textsuperscript{6}}   &	600			& 1230.41     & 51.70				&	1253.31      & 22.90							\\ \hline
							\end{tabular}
					}
				\end{table}
			The number of runs are chosen such that the relative errors (\ref{eq:3}) were under $0.15$ but in case of $10^6$ days the relative error is a bit smaller than the others ($0.11$ vs $0.14$) since with $250$ runs the MAE value was greater than the CI value.
} \fi
			
			
			
% ### REPORT 3 ###
\ifnum\value{debug}=3 {

\chapter{Laboratory \#3}
		\subsection*{Introduction} 
				These experiments involve the use of the English dictionary to check the performance of different data structures when dealing with the membership problem, so answer the question: `'is an element present?''. This is known as set membership problem, actually here it will be considered the approximated set membership problem, since the presence of false positive is accepted.
	 
	 \section{Fingerprinting}	 
	 
			 	\subsection{Number of words in the file}
								The English word dictionary contains $370103$ words.
								
			 	\subsection{Minimum value of bits $b^{exp}$ for no collisions}
						The minimum number of bits necessary to store all the words in a fingerprint table without conflicts is: $b^{exp} = 39$ bits. \\
						To find this value, for each one of the words, the fingerprint is calculated:
						\begin{itemize}
								\item[] Each word is encode using the UTF-8 character encoding, then the MD5 hash is calculated. This MD5 hash returns a 128 bits value but since it is required to have fingerprint length is $b^{exp}$ bits, the MD5 hash value is converted to integer and only the last  $b^{exp}$ bits are taken; this is done with the module operation: \[ fingerprint\_value = word\_hash\_int \: \% \: 2^{b^{exp}} \]
						\end{itemize}
						This operation is repeated for each word and candidate $b^{exp}$ value until the minimum value possible is found. In particular this search is performed using a Binary Search algorithm; to write is a bit more complex than other algorithms but it is faster: for example is faster than an algorithm that checks all numbers between 1 and infinite and stops as soon it find a value of $b^{exp}$ such that any collision is found.
						
				\subsection{Input parameters}
					The input parameters of the simulation are:
					\begin{itemize}
						\item The file containing the English words;
						\item The seed value used to initialize the pseudorandom number generator;
						\item The confidence level used to calculate the confidence interval;
						\item The number of runs: the number of times that we run our simulation. This is done in order to have more accurate;
						\item Number words used for checking the probability of false positive.
						%\item An upper bound for the number of bits. This will be used in a Binary Search algorithm to find the minimum value of bits $b^{exp}$ such that no collisions are experienced.
					\end{itemize}
					 
				\subsection{Output metrics}
					The output metrics of the simulation are:
					\begin{itemize}
							\item The value of $b^{exp}$;
							\item The theoretical value of b: $b^{teo}$;
							\item The storage  memories required for the data structures and the theoretical storage memories;
							\item The probability of false positive using a  $b^{exp}$ fingerprint set.
					\end{itemize}
			
			\subsection{Simulator main data structure}
						There are 2 main data structures used: 
							\begin{itemize}
								\item A python set where to store all the English words;
								\item A python set where to store the fingerprint of each English word.
								\end{itemize}
					
			\subsection{Theoretical number of bit $b^{teo}$}
					It is possible to analytically compute the minimum number of bits given a specified probability \emph{p} using the following formula:
					\[
							b^{teo} \: = \: \log{ \left( - \frac{num\_words^2}{2\times \ln\left({1-p}\right)} \right)}
					\]
					This formula is obtained solving the probability of conflict equation of the Birthday Paradox in function of n (the number of days). In this case the number of days is substituted by the possible length of the fingerprint table($n=2^bits$).
					
			\subsection{Relation $b^{exp}$ and $b^{teo}$}
					The theoretical value and the simulated value are really close: 
					\[
							b^{exp} \: = \: 39; \; 	b^{teo} \: = \: 37 \:  
					\]
			\[ \text{(the decimal value is 36.52), ratio = 39/37 = 1.05} \]
			
			\subsection{Theoretical amount of memory}
			The theoretical memory, in MB, required to store all the words using the two structures, fingerprint table and python set, is:
			\[th\_size\_fp\_table \: = \: num\_words \times \frac{b^{exp}}{8} \times \frac{1}{1024^2}\: = \: 1.72 \text{ MB}\] 
			\[th\_size\_python\_set \: = \: num\_words \times 4.79 \times \frac{1}{1024^2} \: = \: 1.69 \text{ MB}\] 
			where $4.79$ is the average length words of the English dictionary. (1 character = 8 bits) (\href{http://norvig.com/mayzner.html}{reference}).
						
			\subsection{Actual amount of memory}
			To calculate the memory used by an object in python it is possible to use the \emph{asizeof(obj)} from the \emph{pympler} library. This returns the memory in bytes of a given object.
			\begin{center}
					Memory required to store: the fingerprint table: 16.47 MB, for the set of words: 21.46 MB
			\end{center}
			
			\subsection{Probability false positive}
			%Given $b^{exp}$-fingerprint set, 
			There are different ways to calculate the probability of false positive, simulation is one of this. \\
			This `simulation' is performed creating some `fake' words, these are not real word but are integer numbers between [0,n). This is done to simulate the behaviour of the hash function. Given these `fake' word a check is performed to understand if the `word hash' is present in the fingerprint set, if it is, then there is a case of false positive.
			
%			But in this case is not needed since it is possible to calculate the probability of false positive analytically, using the following formula:
%			\begin{center}
%					prob(false\_pos) = number\_words / $2^{b^{exp}}$
%			\end{center}
			
			\subsection{Summary previous results}

				\begin{table}[h!]
				\renewcommand{\arraystretch}{1.35}
				  \resizebox{0.5\textwidth}{!}{%
				\centering
						\begin{tabular}{|l|c|c|c|c|}
						\hline
						\multicolumn{1}{|c|}{\textbf{Storage}} &
						  \multicolumn{1}{l|}{\textbf{\begin{tabular}[c]{@{}l@{}}Bits per \\ fingerprint\end{tabular}}} &
						  \multicolumn{1}{l|}{\textbf{\begin{tabular}[c]{@{}l@{}}Prob. \\ false positive\end{tabular}}} &
						  \multicolumn{1}{l|}{\textbf{\begin{tabular}[c]{@{}l@{}}Min theo. \\ memory (MB)\end{tabular}}} &
						  \multicolumn{1}{l|}{\textbf{\begin{tabular}[c]{@{}l@{}}Actual \\memory (MB)\end{tabular}}} \\ \hline
						\textbf{Word set} &
						  N.A. &
						  0 &
						  1.69 &
						  21.46 \\ \hline
						  
						\textbf{Fingerprint set} &
						  39 &
						  $2.66 \times 10^{- 7}$ &
						  1.72 &
						  16.47 \\ \hline
						\end{tabular}
						}
				\end{table}
				It is possible to see that  the memory required for the python set is much more the memory required to store the fingerprint table. This difference is more evident if the elements to be stored are different in sizes, for example: to store audio files, the memory required for the python set would be very high instead the one for the fingerprint table would not change (as long as we keep the same value of number of bits). \\
				The probabilities of false positive is calculated using $ 5 \times 10^{6}$ number of testing words (words not in the English dictionary) but the probability of false positive for the fingerprint table in some runs were still 0. \\
				Actually, this rare event and would require even an higher number of test words but this would take more time and computational power so in this case it is better to calculate analytically with the following formula:
				\begin{center}
					prob(false\_pos) = number\_words / $2^{39} =  6.732 \times 10^{-7}$
			\end{center}
				
%			\newpage
					
			
			\section{Bit String Array}	 
	 
				\subsection{Input Parameters}
					The input parameters of the simulation are:
					\begin{itemize}
						\item The file containing the English words;
						\item A list with the possible values of the number of bits: [19, 20, 21, 22, 23, 24];
						\item A flag to indicate that we want to use Bit String Array or Bloom Filter. (0:Bit String Array).
					\end{itemize}
					 
				\subsection{Output Metrics}
				The output metrics of the simulation are:
				\begin{itemize}
					\item Number of bits;
					\item The probability of false positive;
					\item The memory occupancy of the bit string array;
					\item The theoretical memory occupancy.
				\end{itemize}
				All these value are stored in a file and each field is tab separated.
			
			\subsection{Main data structures}
						There are 3 main data structures used: 
							\begin{itemize}
								\item A python set to store all the English words;
								\item A bit string array,  from the \emph{bitarray} library, of length $2^{\#bits}$;
								\item A numpy array, of length \#runs, where to store the probability of false positive at each run.
							\end{itemize}
					The bit string array allows to determine if an element is already stored or not with a constant access time of $\mathcal{O}(1)$, since it share the same property of a simple array. \\
					This implementation allows to reduce the storage memory.
					
			\subsection{Probability of false positive in function of bits per fingerprint}
			\includegraphics[width = 0.45\textwidth]{Lab3/BitStringArrayProbabilityFalsePositive.png} \\
			The probability of false positive is calculated as:
				%prob(arr[H(w)]=1|H(w) \not\in arr) \: = \: \frac{\#1s}{total\_length} 
			\begin{equation}\label{eq:1}
						prob(false\_pos) \: = \: \frac{\#1s}{total\_length}
			\end{equation}
			%Where arr is the bit string array, w is the word not in the english vocabulary, $H(\cdot)$ is the hash function and $tota\_length$ is the total size of the bit string array.
			where $\#1s$ is the number of 1s in the bit string array and $total\_length$ is the total size of the bit string array.
					
					\newpage
					
			\subsection{Summary previous results}
			
\begin{table}[h!]
				\renewcommand{\arraystretch}{1.35}
				\resizebox{0.5\textwidth}{!}{%
				\begin{tabular}{|c|c|c|c|c|}
				\hline
				\textbf{Storage} &
				  \textbf{\begin{tabular}[c]{@{}c@{}}Bits\\ fingerprint\end{tabular}} &
				  \textbf{\begin{tabular}[c]{@{}c@{}}Prob.\\ false positive\end{tabular}} &
				  \textbf{\begin{tabular}[c]{@{}c@{}}Min theo.\\ memory (MB)\end{tabular}} &
				  \textbf{\begin{tabular}[c]{@{}c@{}}Actual\\ memory (MB)\end{tabular}} \\ \hline
				\multirow{6}{*}{\textbf{\begin{tabular}[c]{@{}c@{}}Bit String\\ Array\end{tabular}}}
																																	& 19 & 0.506  	& \multirow{6}{*}{0.044} & 0.066  \\ \cline{2-3} \cline{5-5}
                                                     & 20 & 0.297  	&  																							&  0.125  \\ \cline{2-3} \cline{5-5}
                                                    & 21 & 0.162  	& 															 										& 0.25  \\ \cline{2-3} \cline{5-5}
                                                     & 22 & 0.084   &  																								&  0.5  \\ \cline{2-3} \cline{5-5}
                                                     & 23 & 0.043 		&  																								& 1.0  \\ \cline{2-3} \cline{5-5}
                                                      & 24 & 0.022  		&  																							& 2.0 \\ \hline
				\end{tabular}%
				}
\end{table}
\raggedbottom
\noindent
		The theoretical probability is calculated as the memory required to store the number of words using a 1 bit (possible values: True, False) so it does not depend on the number of bit used. Formula:
		\[ theoretical\_memory \: = \: num\_words \times \frac{1}{8} \times \frac{1}{1024^2} \: = \: 0.044 \text{ MB}\] \\
		It is possible to notice that probability of false positive is not as low as the one of fingerprint table but the used memory is much lower.
		
%		\newpage
		
					\section{Bloom Filters}	 
	 				The Bloom Filters are just an extension of the Bit String Array so they share lot of properties and code. \
	 				
				\subsection{Input parameters}
					The input parameters of the simulation are:
					\begin{itemize}
						\item The file containing the English words;
						\item A list with the possible values of the number of bits: [19, 20, 21, 22, 23, 24];
						\item A flag to indicate that we want to use Bit String Array or Bloom Filter. (1:Bloom Filter).
						\item The seed value used to initialize the pseudorandom
number generator;
						\item The confidence level used to calculate the confidence interval;
						\item The number of runs: the number of times that we run our simulation. This is done in order to have more accurate.
					\end{itemize}
					 
			\subsection{Output metrics}
				The output metrics of the simulation are:
				\begin{itemize}
							\item Number of bits;
							\item The number of hashes used.
							\item The lower confidence interval value;
							\item The average probability of false positive;
							\item The upper confidence interval value;
							\item The relative error;
							\item The theoretical probability of false positive;
							\item The memory occupancy of the bit string array;
							\item The theoretical memory occupancy.
				\end{itemize}
				
					\subsection{Main data structures}
					The same as the bit string array. \\
			For Bloom Filters, the time needed to determine if an element is present or not is constant and it has a worst case access time of $\mathcal{O}(k)$, where k is the number of hash functions used.
					
			\subsection{Probability of false positive}
			\includegraphics[width = 0.45\textwidth]{Lab3/BloomFilterProbabilityFalsePositive.png} \\
			The simulation is run using 30000 `fake' words and even with such a lower number the simulated results are quite accurate. \\
			The theoretical value is calculated as:
			\begin{equation} \label{eq:2}
				prob(false\_pos) \: = \: \left( 1 - e^{\frac{k_{opt} \times num\_words}{total\_length}} \right)^{k_{opt}}
			\end{equation}
			where $total\_length$ is the total size of the bloom filter, $k_{opt}$ is the optimal number of hashes to use and is calculated as: 
			
			\begin{equation} \label{eq:3}
					k_{opt} = \frac {2^{\#bits}}{number\_words} \times \ln(2)
			\end{equation}
			\noindent
			Given this value, one should check both upper and lower integer (ceil() and floor()) and pick the best one but usually round to the nearest integer works fine.
			
%			\newpage 
			
			\subsection{Summary previous results}
			
		\begin{table}[h!]
				\renewcommand{\arraystretch}{1.35}
				\resizebox{0.5\textwidth}{!}{%
				\begin{tabular}{|c|c|c|c|c|}
				\hline
				\textbf{Storage} &
				  \textbf{\begin{tabular}[c]{@{}c@{}}Bits\\ fingerprint\end{tabular}} &
				  \textbf{\begin{tabular}[c]{@{}c@{}}Prob.\\ false positive\end{tabular}} &
				  \textbf{\begin{tabular}[c]{@{}c@{}}Min theo.\\ memory (MB)\end{tabular}} &
				  \textbf{\begin{tabular}[c]{@{}c@{}}Actual\\ memory (MB)\end{tabular}} \\ \hline
				\multirow{6}{*}{\textbf{\begin{tabular}[c]{@{}c@{}}Bloom\\ Filter\end{tabular}}}
																																	& 19 & 0.51  	& \multirow{6}{*}{0.044} & 0.066  \\ \cline{2-3} \cline{5-5}
                                                     & 20 & 0.26  	&  																							&  0.132  \\ \cline{2-3} \cline{5-5}
                                                    & 21 & 0.065  	& 															 										& 0.265  \\ \cline{2-3} \cline{5-5}
                                                     & 22 & 0.004   &  																								&  0.531  \\ \cline{2-3} \cline{5-5}
                                                     & 23 & $1.66 \times 10^{-5}$ 		&  								& 1.062  \\ \cline{2-3} \cline{5-5}
                                                      & 24 & 0*  		&  																							& 2.125 \\ \hline
				\end{tabular}%
				}
\end{table}
\noindent
In general it is possible to see that the performance are better than the bit string array's one: since the memory used is almost the same while the probability of false positive is much lower. \\
In general the performance follow the relations:
\begin{center}
	\emph{bloom filters $>$ bit string arrays $>$ fingerprint tables \\
	(with $k_{opt}>1$)} 
\end{center}

\noindent
	*This result is obtained with 30000 words and 2 runs. Since this is a rare event it would require a lot more words.
				
				\subsection{Performances given 1 MB of memory}
				Theoretically, given 1 MB of memory the performance using the different data structures are:
				\begin{itemize}
					\item Word set: the storage is not possible since we may only store, $\frac{1 MB}{1.69 MB}=0.59$, 59\% of the total words; \\
					\item[] For the other structures it is possible to use number of bits equal to $\frac{1024^2 \times 8}{\#words}=22.66$ bits, so approximating this value, in average for each word, only 22 bits:
					\item Fingerprint table: the storage of the whole list of words is always possible but there is a change on the probability of false positive depending on the number of bits used. Using 22 bits the probability of false positive would be $eps=\frac{\#words}{2^{22}}=0.0882$; 
					\item Bit String Array: Using 22 bits the probability of false positive would be $eps=\frac{1}{22}=0.0454$;
					\item Bloom filter: Using 22 bits the probability of false positive would be $eps=\frac{1}{2^{\frac{22}{1.44}}}=2.51 \times 10^{-5}$. 
				\end{itemize}
				

				\begin{table}[h!]
				\centering
				\renewcommand{\arraystretch}{1.35}
				  \resizebox{0.5\textwidth}{!}{%
						\begin{tabular}{|c|l|c|l|l|l|}
						\hline
						\textbf{\begin{tabular}[c]{@{}c@{}}Data\\ Structure\end{tabular}} &
						  \multicolumn{1}{c|}{\textbf{\begin{tabular}[c]{@{}c@{}}Total Storage\\ Formula (bit)\end{tabular}}} &
						  \multirow{4}{*}{} &
						  \textbf{\begin{tabular}[c]{@{}c@{}}Max\\ Num\\ bits\end{tabular}} &
						  \multicolumn{1}{c|}{\textbf{\begin{tabular}[c]{@{}c@{}}Formula solved \\ for Epsilon\end{tabular}}} &
						  \multicolumn{1}{c|}{\textbf{\begin{tabular}[c]{@{}c@{}}Epsilon\\ Value\end{tabular}}} \\ \cline{1-2} \cline{4-6} 
						\textbf{Fingerprint Table} & \multicolumn{1}{c|}{$m\times \log_2{ \frac{m}{eps} }$}     &  & \multirow{3}{*}{22}  & \multicolumn{1}{c|}{$\frac{m}{2^{22}}$}      									& \multicolumn{1}{c|}{0.0882}      																	\\ [5px]  \cline{1-2} \cline{5-6} 
						\textbf{Bit String Array}  	 & \multicolumn{1}{c|}{$m \times \frac{1}{eps}$} 										   &  &                     							 & \multicolumn{1}{c|}{$\frac{1}{22}$}      														& \multicolumn{1}{c|}{0.0454}      																	\\ [5px] \cline{1-2} \cline{5-6} 
						\textbf{Bloom Filter}      	 & \multicolumn{1}{c|}{$m \times 1.44\log_2{\frac{1}{eps}}$} 		&  &                     							 & \multicolumn{1}{c|}{$ \frac{1}{2^{\frac{22}{1.44}}} $} 		& \multicolumn{1}{c|}{$2.51 \times 10^{-5}$} 						\\ [5px] \hline 
						\end{tabular}
						}
				\end{table}
				\noindent
				where \emph{m} is the number of words to be stored. \\
				It is possible to see that the bloom filters would have better performances given 1 MB of space.

				\subsection{(optional)Optimal number of hash functions}
				\includegraphics[width = 0.45\textwidth]{Lab3/BloomFiltersProbabilityFalsePositive_0_6.png} \\
						\includegraphics[width = 0.45\textwidth]{Lab3/BloomFiltersProbabilityFalsePositive36.png} \\
				It is possible to see that the theoretical formula is accurate since the minimum is found where the probability of false positive is lowest.
				
%				\begin{minipage}{0.4\textwidth}
%						\includegraphics[scale=0.45]{Lab3/BloomFiltersProbabilityFalsePositive_0_3.png}
%				\end{minipage}
%				\begin{minipage}{0.4\textwidth}
%				\end{minipage}
				
				\subsection{(optional) Estimation distinct elements in a bloom filter}
				\includegraphics[width = 0.45\textwidth]{Lab3/ComparisonTheoryvsSimulationnumberdistinctWords_11.png} \\
				Given the theoretical formula: 
				\[ theo\_dist\_words = - \frac{n}{k} \ln \left( 1 - \frac{N_1}{n} \right) \]
				\begin{center}
				where: \emph{n} is the bloom filter storage length, \emph{k} is the optimal number of hash functions given a number of bit  (\ref{eq:3}) and $N\_1$ is the actual number of 1s in the bloom filter. 
				\end{center}
				It is possible to see that the simulation is quite accurate with respect to the theoretical formula. This is shown using the relative errors ($ rel\_err=\frac{|theo\_dist\_words-dist\_words|}{dist\_words} $) and these relative errors are low (the axis scale is quite low) so the theoretical formula is accurate. Moreover it can be noted that the relative errors higher number of bits seems to be more constant while for lower values of bits there is more fluctuation.
				
} \fi

% ### REPORT 4 ###
\ifnum\value{debug}=4 {
    
\chapter{Epidemic Models}
\noindent
	In this chapter there will be analysed some epidemic models, in particular SIR models. A SIR model is the population is divided in 3 categories:
	\begin{itemize}
					\item \textbf{Susceptible}: `health' people;
					\item \textbf{Infected}: people who contracted the virus;
					\item \textbf{Recovered}: people who contracted the virus and spent enough days infected and recovered.
				\end{itemize}
	\noindent
	The values used are:
	\begin{itemize}
					\item Population size ($N$) equal to 10000 individuals;
					\item An infection transmission rate, ($\beta$), equal to $0.2 \: day^{-1}$. This refers to the probability that an infected person infects other susceptible people, when some conditions are met: close contact, no protections and so on;
					\item An recovery period equal to 14 days. So a recovery rate ($\gamma$) equal to $\frac{1}{14 \: days}=0.071 \: day^{-1}$.
				\end{itemize}
				
				
	 \section{Numerical SIR Models}
	 						\includegraphics[width = 0.45\textwidth]{Lab6/NumericalSIRModel.png} \\
								The values of the S,I and R variable are calculated as follow:
								\begin{itemize}
								  \item[] \hspace{12mm} $S(t_{k+1}) = S(t_{k}) - \frac{\beta}{N} \: \Delta t \: S(t_{k}) \: I(t_{k})$
									\item[] \hspace{12mm} $I(t_{k+1}) = I(t_{k}) + \frac{\beta}{N} \: \Delta t \: S(t_{k}) \: I(t_{k}) - \gamma \: \Delta t \: I(t_k)$
									\item[] \hspace{12mm} $R(t_{k+1}) = R(t_{k}) + \gamma \: \Delta t \: I(t_k)$
								\end{itemize}
								
								\noindent
								Actually these are the solutions of differential equations. When solving differential equations, it is needed to define the initial condition, in this case:
								\begin{itemize}
								  \item[] \hspace{25mm} $S(t_{0}) = N - 1$
									\item[] \hspace{12mm} $I(t_{0}) = 1$
									\item[] \hspace{12mm} $R(t_{0}) = 0$
								\end{itemize}
								This is the usual condition of the beginning of an epidemic.
								
			The maximum number of infected people is 2823 and it is reached at the day 80. The pandemic ends at day 229 when the number of infected people is less than 1(?), at this day the number of susceptible people are 705 and the number of recovered one is 9294.
			
			
			\section{Simulative SIR Models}
			
			\subsection{Define all the input parameters of the simulator}
					The input parameters of the simulation are:
					\begin{itemize}
						\item The population size;
						\item The total days the simulation will run for;
						\item The transmission rate, 0.2 in this case (as descriptive above);
						\item The infection period, 14 in this case;
						\item The length movement: how much an individual can move;
						\item The number regions in which divide the population;
						\item The contact range: the distance at which Individual can interact;
						\item The seed value used to initialize a pseudorandom number generator;
						\item The confidence level used to calculate the confidence interval;
						\item The number of runs: the number of times that we run our simulation. This is done in order to have more accurate.
					\end{itemize}
					 
				\subsection{Define all the output metrics of the simulator}
				The output metrics of the simulation are:
				\begin{itemize}
					\item For both the maximum number of infected people and the day at which this occurs:
					\begin{itemize}
						\item The average;
						\item The length of the confidence interval;
						\item The relative error;
					\end{itemize}
					\item The $R_t$ value for each day.
				\end{itemize}
			
			\subsection{Define the main data structures for the simulator}
						The main data structures are: 
							\begin{itemize}
								\item A custom class Individual, where to store some variables (the category, the position in the space ,...) and some methods (Move() to move the individual in the space, ...)
								\item A python list with N objects of type Individual;
								\item A python set where to store the index of the Individual that are infected at day d. This avoid to loop over the all population since $I<N$ (I=number of individual infected);
								\item A python dictionary with key, the number of the region, and as value a set with all the susceptible Individuals in that region. This avoid to loop again over the all population since the number of Individuals in a region is much smaller than N ($n=\frac{N}{number\_regions^2}$, assuming uniform distribution)
							\end{itemize}
							The major problem was in checking the distance between individuals since it is a complexity of $\mathcal{O}(N^2)$. So, some optimizations are needed and performed. \\ The main loop works as follow: \\ For each day, first it iterates for infected\_individuals ($\mathcal{O}(I)$), for infected individual it loops over each susceptible Individual in the region where the infected Individual belongs ($\mathcal{O}(n)$) and it checks the distance and if they are close enough if the infection happened. So, the major complexity of for each day is $\mathcal{O}(I \times n)$ and this value is much smaller than $\mathcal{O}(N^2)$, for I and n smaller than N as it usually is during the days. There are also some other complexity like the time in building the python dictionary but the more complex one was to check the distance between the Individuals.
					
					
	 						\includegraphics[width = 0.45\textwidth]{Lab6/SimulativeSIRModel0.png} \\
								The values of the S,I and R variable are calculated as follow:
								\begin{itemize}
								  \item[] \hspace{5cm} $S(t_{k+1}) = S(t_{k}) - new\_infections$
									\item[] \hspace{5cm} $I(t_{k+1}) = I(t_{k}) + new\_infections - new\_recovered$
									\item[] \hspace{5cm} $R(t_{k+1}) = R(t_{k}) + new\_recovered$
									\item[] \hspace{5cm} $R_t(t_{k+1}) = R_t(t_{k}) + ???$
			 					\end{itemize}
			 Actually these SIR formulas are the same formulas as the numerical case but there here is not an analytical formula for $new\_infections$ and $new\_recovered$ but they have to be obtained through simulation.
			 In particular $new\_infections$ is the number of new infections at day $t_k$ and \\
			  $new\_recovered$ is the number of new recovered at day  $t_k$.
			  								
			The maximum number of infected people is 2312.66 with a confidence interval length of 738.86 (relative error = 0.32) and it is reached at the day 207.33 with a confidence interval length of 83.37. The pandemic ends at day 229 when the number of infected people is less than 1(?), at this day the number of susceptible people are 705 and the number of recovered one is 9294.
			 
			 
} \fi
\end{document}